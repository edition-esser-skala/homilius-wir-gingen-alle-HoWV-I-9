\documentclass[tocstyle=ref-genre]{ees}

\shorttitle{Wir gingen alle}

\begin{document}

\eesTitlePage

\eesCriticalReport{
  –    & – & – & \B1 exclusively contains German terms, which are standardized to modern Italian in this edition:
    st. (stärker) → forte;
    gel. (gelinder) → piano;
    sehr stark → fortissimo;
    gehalten, tenuto;
    gedämpft, con sordino;
    ohne Dämpfer, senza sordino;
    etwas stärker -> poco forte;
    etwas gelassen -> poco piano;
    gedämpft -> con sordino
    In all choral movements (1.1, 1.5, 1.11, 1.14, 1.18, 2.5, 2.7), voices for ob 1/2, vl 1/2, vla, and fond have been added by the editor. ob 2 has been emended if necessary.
  1.1  & – & ob 2 & Bars 1f were emended to accomodate the instrument’s range. \\
  1.2  & – & ob 2 & Bars 67f and 93f were emended to accomodate the instrument’s range. \\
       & 13 & S & 5th \eighthNote\ in \B1: a′8 \\
       & 42 & fag 1 & 2nd \halfNote\ in \B1: \flat b4.–b8 \\
       & 91 & ob 2, vl 2, A & \B1 indeed contains \flat a′2.–\flat a′4. \\
  1.3  & 6 & S & 1st \quarterNote\ in \B1: c″4 \\
  1.4  & 38 & vl 1 & 1st \quarterNote\ in \B1: \sharp c″8–b′8 \\
       & 38 & vl 2 & 1st \quarterNote\ in \B1: a′8–\sharp g′8 \\
  1.7  & 4 & vla & 13th \sixteenthNote\ in \B1: f′16 \\
  1.8  & – & vl, vla, fond & Slurs are somewhat ambiguous in \B1, i.e., is is sometimes unclear whether they start at the 1st or 2nd \thirtysecondNote\ and whether they end at the last \thirtysecondNote\ or at the note on the subsequent beat. Here, they have been resolved uniformly.
  1.15 & – & ob 2 & bar 92 emended
       & – & org & In bars 69, 89f, notes in treble ottavo clef have been added by the editor. \\
  2.6  & 8–151 & org & bass figures missing in \B1 \\
       & 36 & vla & bar in \B1: b2. }
}

\eesToc{
  \part{erstertheil}
  \begin{movement}{nunihr}
    \voice[Coro]
    Nun ihr meine Augenlieder\\
    ſehet auf und ſäumt euch nicht\\
    zu erfüllen jetzo wieder\\
    eure Dank und Schuldes Pflicht.\\
    Weil vorhanden iſt die Zeit,\\
    da euch Heil und Seeligkeit\\
    Jeſus Chriſtus hat erworben,\\
    da Er iſt für euch geſtorben.
  \end{movement}

  \begin{movement}{wirgingen}
    \voice[Coro]
    Wir gingen Alle in der Irre wie Schaafe,\\
    ein jeglicher ſahe auf ſeinen Weg:\\
    Aber der Herr warf unſer Aller Sünde auf Ihn.\\
    (\bibleverse{Is}(53:6))
  \end{movement}

  \begin{movement}{bewahredoch}
    \voice[Soprano]
    Bewahre doch, Judäa, dieſes Wort.\\
    Nun iſt Meſſias aufgetreten,\\
    von dem der Mund glaubwürdiger Propheten\\
    ſchon längſt geſprochen hat.\\
    Merk, was er für dich that,\\
    da nach dem ewgen Rath\\
    der Vater Ihn verſandte,\\
    und diesen Sohn zum Heiland dir ernannte.\\
    Dir macht ſein eigner Mund\\
    ſelbſt die Geſchichte kund.\\
    Die Kelter tret ich ganz allein,\\
    kein Sterblicher ſoll mit mir ſeyn!
  \end{movement}

  \begin{movement}{entziehedich}
    \voice[Soprano]
    Entziehe dich den eitlen Freuden,\\
    und ſchau, o Menſch, auf Jeſu Leiden,\\
    ſein Blut fließt um dein Wohlergehn.\\[1ex]
    Dieß ſind des Lammes harte Kriege,\\
    getroſt, dem Heiland ſind die Siege,\\
    der wird zum Leben dich erhöhn.
  \end{movement}

  \begin{movement}{meinheiland}
    \voice[Coro]
    Mein Heiland iſt gegangen\\
    ins Elend mir zu gut,\\
    verrathen und gefangen,\\
    gegeißelt bis aufs Blut,\\
    geſchlagen und verhöhnet,\\
    verſpeiet und verlacht,\\
    mit Dornenkranz gekrönet,\\
    und gar ans Kreuz gebracht.
  \end{movement}

  \begin{movement}{diefeinde}
    \voice[Basso]
    Die Feinde rüſten ſich, mein Heiland, wider dich.\\
    Ich ſeh dein unſchuldsvolles Leben\\
    der Höllen Schrecken ganz umgeben.\\
    Schon jauchzen bey ſich nahende Gefahren,\\
    gefallne und unſeelge Schaaren,\\
    daß muthig, nie verzagt,\\
    dein heilger Fuß dieß alles wagt.\\
    Die Hölle tobt, und Grauſamkeiten eilt ſie\\
    dir, Heilger, zu bereiten.
  \end{movement}

  \begin{movement}{verachtete}
    \voice[Basso]
    Verachtete, verdammte Sünder,\\
    der Finſterniß verfluchte Kinder,\\
    ſchwört eurem Heiland nicht den Tod!\\
    Bald werden Welt und Himmel brechen,\\
    und er wird ſeine Ehre rächen\\
    als Richter und als euer Gott.\\[1ex]
    Dann wird euch vor des Rächers Blitzen\\
    kein tiefenloſer Abgrund ſchützen,\\
    wenn euer Ohr die Donner hört.\\
    Dann wird er euren Thaten lohnen\\
    und keines Böſewichts verſchonen,\\
    der noch ſein heilig Blut entehrt.
  \end{movement}

  \begin{movement}{diekoenige}
    \voice[Coro]
    Die Könige im Lande lehnen ſich auf,\\
    und die Herren rathſchlagen miteinander\\
    wider den Herrn und ſeinen Geſalbten.\\
    (\bibleverse{Ps}(2:2))
  \end{movement}

  \begin{movement}{siebeschliessen}
    \voice[Alto]
    Sie[h], ſie beſchließen einen Rath,\\
    ſie ſuchen dich mit freundlichem Verſtellen,\\
    mein Theureſter, zu fällen.\\
    Was deine Allmacht that,\\
    ſucht ihre Bosheit zu verſtöhren,\\
    und deinen Namen zu entehren.\\
    Dei göttlich Wort, um das die Heilgen bitten,\\
    wird friſch und ſtolz durch ſie beſtritten.
  \end{movement}

  \begin{movement}{jesudeine}
    \voice[Alto]
    Jeſu, deine heilgen Lehren\\
    zu bewahren, zu vermehren,\\
    ſey mein Glück und meine Pflicht.\\[1ex]
    Spottet immer, freche Sünder,\\
    faßt mich, weltgeſinnte Kinder,\\
    ihr müßt alle fürs Gericht!
  \end{movement}

  \begin{movement}{lassmich}
    \voice[Coro]
    Laß mich dein ſeyn und bleiben,\\
    du treuer Gott und Herr.\\
    Von dir laß mich nicht treiben,\\
    halt mich bey reiner Lehr.\\
    Ach Herr, laß mich nicht wanken,\\
    gieb mir Beſtändigkeit,\\
    dafür will ich dir danken\\
    in alle Ewigkeit.
  \end{movement}

  \begin{movement}{washatmessias}
    \voice[Tenore]
    Was hat Meßias denn gethan?\\
    Sie klagen Ihn erbittert an.\\
    Ich höre ſie, die Ungerechten, ſprechen:\\
    Rebellion ſey dein Verbrechen.\\
    Sie nennen dich und deine Freunde\\
    der Juden und der Römer Feinde.\\
    Er ſucht, ſagt der Verwegnen Mund,\\
    durch liſtige Bemühn die Gunſt\\
    des Volks an ſich zu ziehn,\\
    das Regiment von Juda zu vernichten,\\
    und dann ein weltlich Reich ſich aufzurichten.\\
    Hinweg mit Ihm, er ſtört Judäens Ruh.\\
    Sein Tod muß uns vor den Gefahren,\\
    die uns ſchon drohn, allein bewahren.
  \end{movement}

  \begin{movement}{siemoegen}
    \voice[Tenore]
    Sie mögen dich, mein Heiland, immer haſſen,\\
    die Welt und Freunde mögen dich verlaſſen,\\
    mein Glaube hält dich feſt.\\[1ex]
    Sie mögen ſchmähn, ſie mögen fluchen,\\
    wenn alles mich verläßt,\\
    werd ich in dir mein Leben wieder ſuchen.
  \end{movement}

  \begin{movement}{eristder}
    \voice[Coro]
    Er iſt der Weg, das Licht, die Pfort,\\
    die Wahrheit und das Leben,\\
    des Vaters Rath und ewigs Wort,\\
    den Er uns hat gegeben,\\
    zu einem Schutz, daß wir mit Trutz\\
    an Ihn feſt sollen glauben.\\
    Darum uns bald kein Macht noch Gwalt\\
    aus ſeiner Hand wird rauben.
  \end{movement}

  \begin{movement}{dieihrden}
    \voice[Coro]
    Die ihr den Herrn liebet, haſſet das Arge;\\
    der Herr bewahret die Seelen ſeiner Heiligen,\\
    von der Gottloſen Hand wird er ſie erretten.\\
    (\bibleverse{Ps}(97/96:10))
  \end{movement}

  \begin{movement}{ischarioth}
    \voice[Soprano]
    Iſcharioth, der von der Jünger Schaar\\
    auch einer unter Zwölfen war,\\
    begeht die unſeeligſte That,\\
    um wenig Geld den Heiland zu verrathen,\\
    den Herrn der Himmel und der Welt,\\
    der ihn ſo liebreich lehrte,\\
    und als ein Vater väterlich ernährte.\\
    Ich werd Ihn Meiſter nennen,\\
    dieß ſoll das Zeichen ſeyn,\\
    daran ſollt ihr ihn kennen,\\
    der iſts, den greift.\\
    Kaum hat er aufgehört zu beten,\\
    als Judas und die Krieger zu ihm treten,\\
    und die verfluchte Hand\\
    Ihn feſſelte und band.\\
    Schon hat die Rotte ſich verſammelt,\\
    das Urtheil abzufaſſen:\\
    Tödtet Ihn, ſein Leben muß er laſſen.
  \end{movement}

  \begin{movement}{verlasstihn}
    \voice[Soprano]
    Verlaßt Ihn nicht, ihr vielgeliebten Freunde,\\
    bleibt Ihm getreu, verlacht den Stolz der Feinde,\\
    es kämpft kein Menſch, Gott führt den Streit.\\[1ex]
    Und werdet ihr den Heiland nicht verlaſſen,\\
    wird ſeine Hand die Eure liebreich faſſen,\\
    euch leiten zu der Seeligkeit.
  \end{movement}

  \begin{movement}{dessollst}
    \voice[Coro]
    Des ſollſt du herzlich tröſten dich\\
    in aller Noth beſtändiglich,\\
    mein Tod giebt dir das Leben,\\
    daß du vor mir kannſt ewiglich\\
    in Himmels Freuden ſchweben.
  \end{movement}

  \part{zweytertheil}

  \begin{movement}{weilder}
    \voice[Coro]
    Weil der Gottloſe Uebermuth treibet\\
    muß der Elende leiden.\\
    Sie hängen ſich einander an einander\\
    und erdenken böſe Tücke.\\
    (\bibleverse{Ps}(11/10:2))
  \end{movement}

  \begin{movement}{verschonet}
    \voice[Soprano]
    Verſchonet des Gerechten Blut,\\
    ihr unbarmherzgen Richter,\\
    dämpft eures Grimmes Glut,\\
    bedenket was ihr thut.\\
    Beſchwert euch nicht mit Ungerechtigkeiten\\
    und nie erhörten Grauſamkeiten.\\
    Der Mund des Heilands lehret Frieden,\\
    durch ihn wird uns und euch zugleich\\
    des Vaters Himmelreich\\
    durch ſeinen Sohn beſchieden.\\
    Empört euch nicht, der Sohn wird einſt\\
    auf ſeines Vaters Thron den Ungehorſam rächen\\
    und ein unſeelig Urtheil sprechen.
  \end{movement}

  \begin{movement}{baldwirst}
    \voice[Soprano]
    Mein Heiland, bald wirſt du dein Blut vergießen,\\
    bald wird es, Göttlicher, wie Ströme fließen\\
    für mich und meine Seeligkeit.\\[1ex]
    Wohlan, ſo will ichs feſt im Glauben faſſen,\\
    und dich im Tode ſelbſt nie, nie verlaſſen,\\
    du lebſt und bleibſt in Ewigkeit.
  \end{movement}

  \begin{movement}{jesuderdu}
    \voice[Coro]
    Jeſu, der du wareſt tod,\\
    lebeſt nun ohn Ende:\\
    in der letzten Todesnoth\\
    nirgends hin mich wende\\
    als zu dir, der mich verſühnt.\\
    O mein trauter Herre,\\
    gieb mir nur, was du verdient,\\
    mehr ich nicht begehre.
  \end{movement}

  \begin{movement}{nunrettet}
    \voice[Alto]
    Nun rettet dich, mein Heiland, keine Welt,\\
    das Urtheil iſt gefällt,\\
    ſie beſchloſſen dein Verderben,\\
    du ſollſt den Tod der Sünder ſterben.\\
    Kreuzige ruft die verdammte Schaar.\\
    Sein Blut ſey immerdar\\
    auf uns und unſren Kindern.\\
    Und wie getroſt, wie williglich\\
    nimmſt du dein Kreuz auf dich,\\
    mein Heiland und mein Gott!\\
    verachteſt Schande, Schmach und Spott,\\
    und biſt bereit, den Kampf zu kämpfen,\\
    und Sünde, Hölle, Satan, Tod zu dämpfen.
  \end{movement}

  \begin{movement}{jetztgeht}
    \voice[Alto]
    Jetzt geht auf ungebahnten Wegen\\
    mein Heiland ſeinem Tod entgegen\\
    und trägt ſelbſt ſeines Kreuzes Stamm.\\[1ex]
    Dich will ich zu den nahen Leiden\\
    mit allen Gläubigen begleiten,\\
    der Welt zum Heil erwürgtes Lamm.
  \end{movement}

  \begin{movement}{jesussein}
    \voice[Coro]
    Jeſus ſein Kreuz ſelber trägt,\\
    dran man ihn will heften;\\
    Simon, dems auch auferlegt,\\
    trägt mit allen Kräften;\\
    doch gezwungen ſolchs er faßt.\\
    Gieb, Herr, Kraft und Gaben,\\
    ſo will ich ein Theil der Laſt\\
    ungezwungen tragen.
  \end{movement}

  \begin{movement}{}
    \voice[]
  \end{movement}

  \begin{movement}{}
    \voice[]
  \end{movement}

  \begin{movement}{}
    \voice[]
  \end{movement}
}

\eesScore

\end{document}
